% AUTHOR: ASHWIN ABRAHAM

\documentclass{beamer}
\usepackage{xcolor}
\usepackage{algpseudocode}
\usepackage{framed}
\usepackage{listings}
\usepackage{amsmath}
\usepackage{esint}
\usepackage{hyperref}
\hypersetup
{
colorlinks=true,
linkcolor=blue,
filecolor=magenta,      
urlcolor=cyan,
pdftitle={Parameterized verification through view abstraction},
pdfpagemode=FullScreen,
}
\usepackage[utf8]{inputenc}

\usetheme{Madrid}
\usecolortheme{default}


\title[] 
{\textbf{CS 228 (M) - Logic in CS}\\Tutorial I - Solutions}


\author[Ashwin Abraham] 
{Ashwin Abraham}

\institute[] 
{
    IIT Bombay
}

\date[2023]
{15th August, 2023}



\begin{document}
    \frame{\titlepage}

    \begin{frame}
        \frametitle{Table of Contents}
        \tableofcontents    
    \end{frame}

    \section{Question 1}
    {
        \begin{frame}{Question 1}
            We know that all people involved in this story are either \textit{Angels} or \textit{Normals}, since there are no \textit{Vampires} on the island currently. We can therefore ignore the \textit{Vampires}.

            Let $Ang(x)$ represent the proposition that person $x$ is an \textit{Angel} and $Norm(x)$ represent the proposition that person $x$ is a \textit{Normal}. For every person $x$ we are concerned with, if they are not an \textit{Angel}, they are a \textit{Normal}, and vice versa. Therefore, we have $$Norm(x) \equiv \neg Ang(x)$$%Let us therefore work with $Ang(x)$.
        \end{frame}
        \begin{frame}{Question 1}
            From the question, we know that $A$ and $B$ are from different tribes, therefore $$Ang(A) \iff \neg Ang(B)$$.

            Let $P$ represent $A$'s answer to our question ("yes" corresponding to \textbf{true} and "no" corresponding to \textbf{false}). We know that if $A$ is an \textit{Angel} then his answer is truthful, ie we have $$Ang(A) \implies (P \iff Norm(B))$$rewriting $Norm(B)$ as $\neg Ang(B)$ and including the previous constraint we get our final constraint as:
            \begin{equation*}
                \boxed{\left[Ang(A) \iff \neg Ang(B)\right] \land \left[Ang(A) \implies \left[P \iff \neg Ang(B)\right]\right]}
            \end{equation*}
        \end{frame}
        \begin{frame}{Question 1}
            The last thing we need to use to solve this puzzle is the fact that after we heard $A$'s answer, we were able to figure out who is who, ie after we found the value of $P$, we were also able to find a \textbf{unique solution} for $Ang(A)$ and $Ang(B)$.

            You can verify that the truth values of $P, Ang(A), Ang(B)$ that satisfy the constraint are:
            \begin{enumerate}
                \item $(P, Ang(A), Ang(B)) = (T, T, F)$
                \item $(P, Ang(A), Ang(B)) = (T, F, T)$
                \item $(P, Ang(A), Ang(B)) = (F, F, T)$
            \end{enumerate}
            Clearly, the solution for $(Ang(A), Ang(B))$ is unique iff $P$ is \textbf{false}, and the unique solution is $(P, Ang(A), Ang(B)) = (F, F, T)$.

            Therefore, $A$ is a \textit{Normal}, $B$ is an $\textit{Angel}$ and when asked, $A$ (truthfully!) told us that $B$ was not a \textit{Normal}.
        \end{frame}
    }

    \section{Question 2}
    {
        \begin{frame}{Question 2}
            Let's say the people in the hall are numbered from $0 \dots n - 1$, where you are person $0$. Let $L_{i}$ represent the proposition that person $i$ is a liar. $L_{0}$ is known to us, and we also know that the total number of liars in the hall is even. This can be written as\footnote{It can be shown that $\bigoplus_{i = 1}^{n} x_{i}$ is true precisely when an odd number of $x_{i}$ are true. I am also abusing notation slightly by using $=$ instead of $\iff$}:
            \begin{equation*}
                L_{0} = \bigoplus\limits_{i = 1}^{n - 1} L_{i}
            \end{equation*}

            Let $Q_{i}$ represent the response of the $i^{th}$ person to our question. If we knew both $L_{i}$ and $Q_{i}$ for any person, then what we require is just $L_{i} \oplus Q_{i}$. Note that $L_{i} \oplus Q_{i}$ must be the \textbf{same} for every person $i$. Call this quantity $P$.
            \begin{equation*}
                P = L_{i} \oplus Q_{i}, \forall i \in \{0 \dots n - 1\}
            \end{equation*}
            $P$ is what we want to find.
        \end{frame}
        \begin{frame}{Question 2}
            However, we do not know both $L_{i}$ and $Q_{i}$ for any given person $i$. We know $L_{0}$ and can choose some $i \in \{1 \dots n - 1\}$ and find the corresponding $L_{i}$. We can find $Q_{j}$ for any $j \neq 0, i$ where $i$ was the number chosen earlier.
            
            Our strategy is as follows:

            If $n$ is even, we do not ask anyone if they are a liar but instead ask everyone the question, and obtain $Q_{i}, \forall i \in \{1 \dots n - 1\}$.
            Now, applying $\oplus$ on the last equation for all $i \in \{1 \dots n - 1\}$, we get
            \begin{equation*}
                P = \left(\bigoplus\limits_{i = 1}^{n - 1} L_{i}\right) \oplus \left(\bigoplus\limits_{i = 1}^{n - 1} Q_{i}\right) = L_{0} \oplus \left(\bigoplus\limits_{i = 1}^{n - 1} Q_{i}\right)
            \end{equation*}
            and we can hence obtain $P$.
        \end{frame}
        \begin{frame}{Question 2}
            If $n$ is odd, we pick an arbitrary $j \in \{1 \dots n - 1\}$ and ask person $j$ if she is a liar. We ask every other person the question. Therefore, we know $Q_{j}$ $\forall j \in \{1 \dots n - 1\} - \{j\}$ and $L_{j}$.
            Now,
            \begin{equation*}
                L_{0} \oplus L_{j} = \bigoplus\limits_{\substack{i = 1\\i \neq j}}^{n - 1} L_{i}
            \end{equation*}
            Therefore, on similar lines as before, we get
            \begin{equation*}
                P = L_{0} \oplus L_{j} \oplus \left(\bigoplus\limits_{\substack{i = 1\\i \neq j}}^{n - 1} Q_{i}\right)
            \end{equation*}
        \end{frame}
    }
    \section{Question 3}
    {
        \begin{frame}{Question 3}
            Note that if $B$ is a knight, then $A$ must be a knave, which makes her statement that $B$ is a knight false, which contradicts the fact that $B$ is a knight. Therefore $B$ is not a knight. This means $A$'s statement is false, which means $A$ is also not a knight. So both are either normals or knaves, and $A$'s statement is false. Let us take some cases now:
            \begin{enumerate}
                \item If $A$ is a knave, then $B$'s statement is true, which means $B$ must be a normal. This means one of them ($B$) told the truth but is not a knight.
                \item If $B$ is a knave, then $A$ must be a normal. This means one of them ($A$) told a lie but is not a knave.
                \item If both $A$ and $B$ are normals, then both their statements are false, which means one of them (either) told a lie but is not a knave.
            \end{enumerate}
            Therefore, by cases, we can say that one of them told the truth but is not a knight, or that one of them told a lie but is not a knave.

            This natural language proof can also be directly translated into a formal proof.
        \end{frame}
    }
    \section{Question 4}
    {
        \begin{frame}{Question 4}
            Let $P(i, j)$ represent the proposition that the $i^{th}$ pigeon is sitting in the $j^{th}$ hole, where $i \in \{1\dots n + 1\}$ and $j \in \{1\dots n\}$.

            The Pigeonhole principle states that, if there are $n + 1$ pigeons and $n$ holes, and every pigeon sits in exactly one hole, then there is a hole occupied by more than one pigeon. To convert this into a PL formula, let us convert each side of the implication into PL first.

            Every pigeon sits in at least one hole can be expressed in PL as:
            \begin{equation*}
                \bigwedge\limits_{i = 1}^{n + 1} \bigvee\limits_{j = 1}^{n} P(i, j)
            \end{equation*}
            Here the inner disjunction refers to the $i^{th}$ pigeon sitting in some hole, and the outer conjuction makes it so that every pigeon must sit in some hole. Call this condition $F$.
        \end{frame}
        \begin{frame}{Question 4}
            We also need no pigeon to sit in multiple holes. Say pigeon $i$ sits in holes $j$ and $k$ with $j < k$. The formula $P(i, j) \land P(i, k)$ represents this scenario. There exists a pigeon sitting in multiple holes therefore becomes:
            \begin{equation*}
                \bigvee\limits_{i = 1}^{n + 1} \bigvee\limits_{\substack{j, k = 1\\j < k}}^{n} \left(P(i, j) \land P(i, k)\right)
            \end{equation*}
            Here, the inner disjunction refers to the $i^{th}$ pigeon sitting in multiple holes and the outer disjunction refers to there existing a pigeon sitting in multiple holes.
            
            Negating this, we get the condition for no pigeon to sit in multiple holes:
            \begin{equation*}
                \bigwedge\limits_{i = 1}^{n + 1} \bigwedge\limits_{\substack{j, k = 1\\j < k}}^{n} \left(\neg P(i, j) \lor \neg P(i, k)\right)
            \end{equation*}
            Call this condition $G$.
        \end{frame}
        \begin{frame}{Question 4}
            Now, say hole $k$ is occupied by pigeons $i$ and $j$ with $i < j$. We then have $P(i, k) \land P(j, k)$. There exists a hole occupied by more than one pigeon therefore becomes:
            \begin{equation*}
                \bigvee\limits_{k = 1}^{n}\bigvee\limits_{\substack{i, j = 1\\i < j}}^{n + 1} \left(P(i, k) \land P(j, k)\right)
            \end{equation*}
            Here, the inner disjunction refers to the $k^{th}$ hole being occupied by more than one pigeon and the outer disjunction refers to there existing a hole occupied by multiple pigeons. Call this condition $H$.
            
            The Pigeonhole Principle therefore becomes:
            \begin{equation*}
                F \land G \implies H
            \end{equation*}
        \end{frame}
    }
    \section{Question 5}
    {
        \begin{frame}{Question 5}
            Proof:         
            \begin{enumerate}
                \item $\{(p \implies q) \implies q, q \implies p, \neg p, p\} \vdash p$ (Assumption)
                \item $\{(p \implies q) \implies q, q \implies p, \neg p, p\} \vdash \neg p$ (Assumption)
                \item $\{(p \implies q) \implies q, q \implies p, \neg p, p\} \vdash \bot$ ($\bot$ introduction on 1, 2)
                \item $\{(p \implies q) \implies q, q \implies p, \neg p, p\} \vdash q$ ($\bot$ elimination on 3)
                \item $\{(p \implies q) \implies q, q \implies p, \neg p\} \vdash p \implies q$ ($\implies$ intro on 4)
                \item $\{(p \implies q) \implies q, q \implies p, \neg p\} \vdash (p \implies q) \implies q$ (Ass.)
                \item $\{(p \implies q) \implies q, q \implies p, \neg p\} \vdash q$ (Modus Ponens on 5, 6)
                \item $\{(p \implies q) \implies q, q \implies p, \neg p\} \vdash q \implies p$ (Assumption)
                \item $\{(p \implies q) \implies q, q \implies p, \neg p\} \vdash p$ (Modus Ponens on 7, 8)
                \item $\{(p \implies q) \implies q, q \implies p, \neg p\} \vdash \neg p$ (Assumption)
                \item $\{(p \implies q) \implies q, q \implies p, \neg p\} \vdash \bot$ ($\bot$ introduction on 9, 10)
                \item $\{(p \implies q) \implies q, q \implies p\} \vdash \neg \neg p$ ($\neg$ introduction on 11)
                \item $\{(p \implies q) \implies q, q \implies p\} \vdash p$ ($\neg \neg$ elimination on 12)
                \item $\{(p \implies q) \implies q\} \vdash (q \implies p) \implies p$ ($\implies$ intro on 13)
                \item $\emptyset \vdash \left[(p \implies q) \implies q\right] \implies \left[(q \implies p) \implies p\right]$ ($\implies$ intr on 14)
            \end{enumerate}
            $\qed$
        \end{frame}
    }
    \section{Question 6}
    {
        \begin{frame}{Question 6}
            We will use the following proof rule (which has often been used implicitly in the slides) along with the proof rules mentioned in the slides (Call it \textit{Monotonicity}). This rule follows as any proof valid for $\mathcal{H}$ will also be valid for any superset of it.
            \begin{equation*}
                \mathcal{H} \subseteq \mathcal{H}' \implies \frac{\mathcal{H} \vdash \varphi}{\mathcal{H}' \vdash \varphi}
            \end{equation*}

            Proof:
            \begin{enumerate}
                \item $\mathcal{H} \vdash A \implies B$ (Premise)
                \item $\mathcal{H} \vdash C \lor A$ (Premise)
                \item $\mathcal{H} \cup \{C\} \vdash C$ (Assumption)
                \item $\mathcal{H} \cup \{C\} \vdash B \lor C$ ($\lor$ introduction on 3)
                \item $\mathcal{H} \cup \{A\} \vdash A$ (Assumption)
                \item $\mathcal{H} \cup \{A\} \vdash A \implies B$ (Monotonicity on 1)
                \item $\mathcal{H} \cup \{A\} \vdash B$ (Modus Ponens on 5, 6)
                \item $\mathcal{H} \cup \{A\} \vdash B \lor C$ ($\lor$ introduction on 7)
                \item $\mathcal{H} \vdash B \lor C$ ($\lor$ elimination on 2, 4, 8)
            \end{enumerate}
            $\qed$
        \end{frame}
    }
    \section{Question 7}
    {
        \begin{frame}{Question 7}
            Proof:
            \begin{enumerate}
                \item $\mathcal{H} \vdash A \implies C$ (Premise)
                \item $\mathcal{H} \vdash B \implies C$ (Premise)
                \item $\mathcal{H} \cup \{A \lor B\} \vdash A \lor B$ (Assumption)
                \item $\mathcal{H} \cup \{A \lor B, A\} \vdash A$ (Assumption)
                \item $\mathcal{H} \cup \{A \lor B, A\} \vdash A \implies C$ (Monotonicity on 1)
                \item $\mathcal{H} \cup \{A \lor B, A\} \vdash C$ (Modus Ponens on 4, 5)
                \item $\mathcal{H} \cup \{A \lor B, B\} \vdash B$ (Assumption)
                \item $\mathcal{H} \cup \{A \lor B, B\} \vdash B \implies C$ (Monotonicity on 2)
                \item $\mathcal{H} \cup \{A \lor B, B\} \vdash C$ (Modus Ponens on 7, 8)
                \item $\mathcal{H} \cup \{A \lor B\} \vdash C$ ($\lor$ elimination on 3, 6, 9)
                \item $\mathcal{H} \vdash (A \lor B) \implies C$ ($\implies$ intro on 10)
            \end{enumerate}
            $\qed$
        \end{frame}
    }
    \section{Question 8}
    {
        \begin{frame}{Question 8}
            We abuse notation slightly by using $\mathcal{L}$ to also refer to the axiom set. For any formulae $A$, $B$, $C$, we have $\neg \left(A \lor B\right) \lor (B \lor C) \in \mathcal{L}$. We also assume that the proof rules of $\bot$, $\lor$ and $\neg$ are still present in this system.

            We have to show that for any formula $F$, $\mathcal{L} \vdash F$.

            We choose $A = \neg F, B = \bot, C = F$.

            Proof:
            \begin{enumerate}
                \item $\mathcal{L} \vdash \neg (\neg F \lor \bot) \lor (\bot \lor F)$ (Premise)
                \item $\mathcal{L} \cup \{\neg (\neg F \lor \bot), \neg F\} \vdash \neg F$ (Assumption)
                \item $\mathcal{L} \cup \{\neg (\neg F \lor \bot), \neg F\} \vdash \neg F \lor \bot$ ($\lor$ introduction on 3)
                \item $\mathcal{L} \cup \{\neg (\neg F \lor \bot), \neg F\} \vdash \neg (\neg F \lor \bot)$ (Assumption)
                \item $\mathcal{L} \cup \{\neg (\neg F \lor \bot), \neg F\} \vdash \bot$ ($\bot$ introduction on 4)
                \item $\mathcal{L} \cup \{\neg (\neg F \lor \bot)\} \vdash \neg \neg F$ ($\neg$ introduction on 5)
                \item $\mathcal{L} \cup \{\neg (\neg F \lor \bot)\} \vdash F$ ($\neg \neg$ elimination on 6)
            \end{enumerate}
        \end{frame}
        \begin{frame}{Question 8}
            \begin{enumerate}
                \setcounter{enumi}{7}
                \item $\mathcal{L} \cup \{\bot \lor F\} \vdash \bot \lor F$ (Assumption)
                \item $\mathcal{L} \cup \{\bot \lor F, \bot\} \vdash \bot$ (Assumption)
                \item $\mathcal{L} \cup \{\bot \lor F, \bot \} \vdash F$ ($\bot$ elimination on 9)
                \item $\mathcal{L} \cup \{\bot \lor F, F\} \vdash F$ (Assumption)
                \item $\mathcal{L} \cup \{\bot \lor F\} \vdash F$ ($\lor$ elimination on 8, 10, 11)
                \item $\mathcal{L} \vdash F$ ($\lor$ elimination on 1, 7, 12)
            \end{enumerate}
            $\qed$
        \end{frame}
    }
\end{document}